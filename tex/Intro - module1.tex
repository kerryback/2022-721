\documentclass[english,12pt]{amsart}
\RequirePackage{geometry,amsmath,graphicx,babel}

%%%%%%%%%%%%%%%%%%%%%%%%%%%%%%%%%%%%%%%%%%%%%%%%%%%%%%%%%%%%%%%%%%%%%%%%%%%%%%%%%%%%%%%%%%

\geometry{verbose,letterpaper,tmargin=1in,bmargin=1in,lmargin=1.25in,rmargin=1.25in,headheight=0.5in,footskip=0.5in}

\setlength{\parskip}{\bigskipamount}

\setlength{\parindent}{0pt}


\usepackage{hyperref}

%%%%%%%%%%%%%%%%%%%%%%%%%%%%%%%%%%%%%%%%%%%%%%%%%%%%%%%%%%%%%%%%%%%%%%%%%%%%

\begin{document}

\begin{center}
    Module 1\\
    BUSI 721
\end{center}

\vskip 2\baselineskip

How much do you need to save each year to provide for a comfortable retirement?  How do you adjust your savings plan for inflation?  What would the mortgage payments be on that house you want to buy?  Given what you can afford to pay each month, how much can you afford to borrow?  We'll answer these questions and more in this module.  We'll track retirement savings accounts or loan accounts and see how to calculate future and present values.  We'll cover the interest-rate math and also introduce code to perform the calculations.

\end{document}